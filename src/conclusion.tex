\chapter*{Zakończenie}
\label{chap:zakonczenie}
\addcontentsline{toc}{chapter}{Zakończenie}


Celem pracy było opisanie czytelnikowi czym jest technologia VR,
omówienie procesu powstania emocji oraz narzędzi wykorzystywanych do ich wywołania w
grach oraz zbadanie, jakimi zdolnościami do wywołania konkretnych emocji cechują się gry VR oraz gry wideo.
    
W pierwszym rozdziale pracy zdefiniowano technologię wirtualnej rzeczywistości jako połączenie sprzętu i oprogramowania, które tworzy całkowicie nowe, cyfrowe symulacje. VR składa się z trzech głównych części jakimi są:  immersja, wyobraźnia, interakcja. W tym rozdziale została także opisana historia VR: od początków sięgających ściennych malowideł po lata współczesne, skupiające się głównie na rozrywce VR, ale także zostały podane prognozy na przyszłość mówiące o jej nowych zastosowaniach.

W kolejnym rozdziale przybliżono pojęcie emocji jako stanu umysłu, któremu towarzyszą zmiany somatyczne, akty ekspresji i działania. Dokonano  klasyfikacji emocji na funkcjonalne jak i niefunkcjonalne oraz omówiono proces powstania emocji dzieląc go na 4 składowe jakimi są: 
``ocena poznawcza'', ``wartościowanie kontekstowe'', ``gotowość do działania'' i ``zmiany psychologiczne, ekspresja, działanie''.

Trzeci rozdział poświęcony został na ukazanie związku emocji z grami. W tym celu przywołano 6 podstawowych emocji jakimi są: gniew, strach, wstręt, radość, smutek oraz zaskoczenie. Zdefiniowano grę jako aktywne medium, które jest rozróżnialne od pasywnych (np. filmów) dzięki swojej głównej właściwości jaką jest interaktywność. Wymieniono charakteryzujące gry atrybuty, które pozwalają na wywołanie emocji: użycie awatara, interaktywność, kontrolę, poczucie sprawczości, symulowane akcje, interaktywny dyskomfort, wyzwanie, doświadczenie społeczne, zaangażowanie ciała.

W ostatnim rozdziale zaprezentowano i przedyskutowano wyniki badań. Przeprowadzone badania pokazały, że preferencje graczy w stosunku do ulubionej platformy znacznie się różnią w zależności od wieku - najmłodsi i najstarsi gracze preferują rozrywkę w trybie VR, natomiast osoby w wieku od 16 do 40 lat wybierają rozrywkę w formie gier wideo. Gracze wybierają najczęściej gry zręcznościowe, sportowe oraz symulacyjne, a ich ulubioną częścią rozgrywki jest najczęściej element wyzwania, rywalizacji i osiągnięcia sukcesu. Rozgrywka graczy różni się znacząco w zależności od wybranej platformy. Osoby korzystające z gier VR odbywają krótkie sesje o małej częstotliwości z powodu negatywnych skutków, jakie powoduje częste i długie korzystanie z gogli wirtualnej rzeczywistości, natomiast osoby korzystające z gier wideo grają częściej i dłużej ze względu na bardziej relaksujący charakter rozgrywek. Pomimo że obie platformy mają dobre predyspozycje do wywołania emocji wśród graczy, to dzięki większej zdolności do pochłonięcia uwagi gracza, lepiej radzi sobie z tym technologia VR. Zauważalny jest także związek pomiędzy platformą a rodzajem emocji, które wywołuje. VR znacząco lepiej radzi sobie z wywołaniem zaskoczenia oraz strachu w porównaniu do gier wideo, które z kolei lepiej radzą sobie z wywołaniem smutku u graczy. Predyspozycje platform do wywołania radości i wstrętu są podobne, natomiast większość badanych wybrało gry wideo jako platformę mającą większą zdolność do wywołania gniewu. Analizując dane wywnioskować można, że chociaż nie wszystkie emocje są nacechowane pozytywnie, są one pożądane przez graczy. Tymi, które są najważniejsze dla graczy to kolejno: radość, zaskoczenie i strach, a te najmniej pożądane to gniew wstręt oraz smutek. 

Gry VR przechodzą powolną transformacje i w przyszłości spodziewać się będzie
można, że zostaną rywalem gier wideo oraz stanowić będą coraz to większą część rynku
gier. Intensywność wywołanych emocji, ich nacechowanie oraz sposób rozgrywki znacząco się różni w przypadku obu platform stąd porównywanie, która z nich jest lepsza wydaje się być bezcelowe ponieważ każda z nich ma na celu dostarczenie innych wrażeń użytkownikowi. Wnioski takie można wysunąć również na podstawie odpowiedzi ankietowanych, co potwierdza, że technologia VR jest pożądana, nie tylko jako forma rozrywki, ale także w dziedzinach naukowych. Pomocne w tej kwestii jest finansowe ugruntowanie technologii VR i rynku, na
którym głównie bazuje. Przewidywania te jednak mogą również odwoływać się do gier
wideo, które rozwijają się w ogromnym tempie i spotykają się z niezwykle pozytywny
odbiorem wśród coraz to większej populacji graczy. Możemy zatem stwierdzić, że niezależnie od formy, gry VR i
gry wideo są niezastąpionym sposobem rozrywki, spędzania wolnego czasu oraz przeżywania
wirtualnych przygód.


% Technologia VR pozwala na wywołanie silnych i intensywnych emocji u swoich
% użytkowników, lecz niesie to za sobą także negatywne skutki, stad wiele osób wybiera
% klasyczne gry wideo. Zestawiając obie formy użytkowania gier (tj. wideo i VR) oraz biorąc
% pod uwagę emocjonalność gracza możemy wyodrębnić wiele zalet i wad, które wpływają na
% wywoływanie danego afektu u użytkownika. Wybór platformy zależy w głównej mierze od
% tego, jakiego rodzaju rozrywki oczekuje użytkownik. Intensywność wywołanych emocji, ich
% nacechowanie oraz sposób rozgrywki znacząco się różni w przypadku obu platform stąd
% porównywanie, która z nich jest lepsza wydaje się być bezcelowe. Gry VR często występują
% w nieskomplikowanych formach, aniżeli jako pełnoprawny produkt konsumencki. Jednym z
% wyjątków jest wymieniony wielokrotnie w pracy ‘Half Life Alyx’, który można nazwać
% faktyczną pełnoprawną wersją gry VR. Jak zostało opisane w tekście, każda forma wpływa na
% inny rodzaj emocji. Tak samo postrzegają to ankietowani, wg których znaczną przewagę nad
% wywoływaniem m.in. zaskoczenia ma technologia VR.