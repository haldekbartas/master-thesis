\chapter*{Wstęp}
\label{chap:wstep}
\addcontentsline{toc}{chapter}{Wstęp}

Komputery nie służą już człowiekowi tylko jako narzędzie pracy z wysoką mocą obliczeniową, ale także dostarczają rozrywki wysokiej jakości bez potrzeby wychodzenia z domu. Gry pozwalają na aktywne doświadczenie, które może wyzwolić więcej emocji, a co za tym idzie więcej rozrywki niż media pasywne, dzięki narzędziom wykorzystywanym przez ich projektantów i twórców. 

Ludzie doświadczają rozmaitych emocji podczas grania w gry i jest to główny powód, który przyciąga miliardy graczy przed monitory w ciągu każdego roku. Gry wywołują różne stany emocjonalne u odbiorcy - różne typy gier są więc popularne wśród różnych odbiorców. Nie ma uniwersalnego doświadczenia, które zaspokoiłoby potrzeby każdego gracza, ale wszystkie pozycje mają wspólny mianownik, a mianowicie wywołanie konkretnego stanu u odbiorcy. Żadne inne medium nie oferuje takiej mocy transformacji jednostki na poziomie indywidualnym i społecznym, co jest możliwe dzięki dużej immersji przy równoczesnej możliwości wcielenia się w postać. Projektanci gier nadal przesuwają granice, pozwalając na coraz lepszą eksplorację świata oraz postaci, w które wciela się gracz.

Celem niniejszej pracy jest przybliżenie czytelnikowi czym jest technologia wirtualnej rzeczywistości (ang. Virtual Reality - VR), opisanie procesu powstania emocji oraz narzędzi wykorzystywanych do ich wywołania w grach oraz zbadanie jakimi zdolnościami do wywołania konkretnych emocji cechują się gry VR i gry wideo.  W toku badań własnych autora pracy udzielone zostaną odpowiedzi na następujące pytania badawcze: 

\begin{enumerate}
   \item Jaka jest charakterystyka graczy preferujących gry wideo, a jaka preferujących gry VR?
   \item Czym charakteryzuje się rozgrywka w gry typu VR, a czym w gry wideo?
   \item Czy emocje w grach są pożądane z perspektywy graczy oraz jaki jest związek pomiędzy platformą a wywołanymi emocjami?
\end{enumerate}

W tym celu przeprowadzono badania za pomocą kwestionariusza ankiety wśród osób mających styczność z grami wideo oraz technologią wirtualnej rzeczywistości.

Praca składa się z czterech rozdziałów. W pierwszym rozdziale scharakteryzowana zostaje technologia wirtualnej rzeczywistości, przybliżona zostaje jej historia oraz ewolucja na przestrzeni lat. Drugi rozdział definiuje pojęcie emocji, przedstawia jej funkcje oraz opisuje przebieg jej powstania u człowieka poprzez omówienie całego procesu: od oceny poznawczej aż po wykonanie działania. W kolejnym rozdziale zaprezentowano klasyfikację emocji oraz ukazanie jej związku z grami. Szczegółowo opisane zostają narzędzia wykorzystywane do wywołania emocji w grach oraz różnice między grami a innymi mediami pod względem wywołania emocji. W rozdziale czwartym zaprezentowano wyniki badań autora pracy. Scharakteryzowana została grupa badawcza, przedstawiono i przedyskutowano wyniki badań oraz udzielono odpowiedzi na sformułowane powyżej pytania badawcze.